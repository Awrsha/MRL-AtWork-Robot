\section{\RCAW in a Nutshell}\label{sec:at_work_nushell}
\RCAW is a competition in \RC that targets the use of robots in work-related scenarios. \RCAW utilizes proven ideas and concepts from \RC competitions to tackle open research challenges in industrial and service robotics. With the introduction of this new event, \RC opens up to communities researching both classical and innovative robotics scenarios with very high relevance for the robotics industry. 
\par
Examples for the work-related scenarios targeted by \RCAW include

\begin{itemize}
	\item loading and/or unloading of containers with/of objects with the same 	or different size,
	\item pickup or delivery of parts from/to structured storages and/or 				unstructured heaps,
	\item operation of machines, including pressing buttons, opening/closing 			doors and drawers, and similar operations with underspecified or unknown 			kinematics,
	\item flexible planning and dynamic scheduling of production processes 			involving multiple agents (humans, robots, and machines),
	\item cooperative assembly of non-trivial objects, with other robots 				and/or humans,
	\item cooperative collection of objects over spatially widely  distributed 	areas, and
	\item cooperative transportation of objects (robots with robots, robots 			with humans).
\end{itemize}

The \RCAW scenarios target difficult, mostly unsolved problems in robotics, artificial intelligence, and advanced computer science, in particular in perception, path planning and motion planning, mobile manipulation, planning and scheduling, learning and adaptivity, and probabilistic modeling, to name just a few. Furthermore, \RCAW scenarios may also address problems for which solutions require the use and integration of semantic web technology, RFID technology, or advanced computational geometry.
\par

Solutions to the problems posed by \RCAW require sophisticated and innovative approaches and methods and their effective integration. The scenarios are defined such that the problems are sufficiently general and independent of particular industrial applications, but also sufficiently close to real application problems that the solutions can be adapted to particular application problems with reasonable effort.
\par

\clearpage

A \RCAW competition has only recently become a feasible idea for several reasons: The arrival of new, small, and flexible robot systems for mobile manipulation allow more university-based research labs to perform research in the above-mentioned areas. Advances and a revived interest in the use of simulation technology in robotics enable research groups to perform serious research without having a full set of costly robotics and automation equipment available.
\par

The robotics and automation industry is recently shifting its attention towards robotics scenarios involving the integration of mobility and manipulation, larger-scale integration of service robotics and industrial robotics, cohabitation of robots and humans, and cooperation of multiple robots and/or humans. Last but not least, there is a huge interest by funding agencies and professional societies in well-designed and professionally performed benchmarks for industry-relevant robotics tasks. \RCAW is designed as an instrument to serve all these needs.
\par


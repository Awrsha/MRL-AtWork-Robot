% !TEX root = ../Rulebook.tex

\chapter{Summary of Changes}

\begin{comment}
This chapter provides an overview for experienced teams that know the rules and just need an update on what is new for the specific year. 
All new teams are strongly advised to read the whole rule book thoroughly.
\end{comment}

\section{Season 2024}

For the Season 2024, we mainly added some clarifications regarding onsite events and the usage of ATTCs.

\subsection{Open Source Award}

We added an open source award which should encourage share of code and knowledge between teams with the intention to improve the league community and simplify the initial steps to participate (see \ref{sec: osa}).

\subsection{Clarifications}

We added some clarifications regarding:

\begin{itemize}
	\item Arbitrary Surfaces and objects sinking into them
	\item Default table size and allowed margin for onsite constructions
	\item Placing multiple objects into the same container
	\item The maximum size for decoy objects to avoid damage on manipulators
\end{itemize}

\subsection{Teamleader Meetings}

We added timeslots for teamleader meetings to the official onsite schedule.
They have always been part of onsite events but were a bit difficult to organize without official requirements 
with the recently increasing number of participating teams.

\subsection{April Tagged Object - Details}

We've added some detailed requirements for the usage of April Tagged Objects,
mainly targeting better usability and handling for onsite events.
We also removed some bonus points for manipulation of ATTCs as we felt that they were unjustified
considering the much lower technical hurdle.

To ensure that we can actually replace a regular task with the simplified ATT version,
teams wanting to use that option are required to provide two complete sets of prepared objects for onsite events.

\section{Season 2023}

\subsection{Restructuring of Benchmark Tests}

The specialized tasks Precise Placement and Rotating Table have been integrated into the new Advanced Transportation Tasks and the Final.
The standalone tests have therefore been removed from the competition schedule, effectively reducing the number of total tests by one.

\subsection{Linear Increase of Complexity}

The elements included in the different tests have been adjusted to form a more linear increase of overall task complexity over the course of the competition. This should allow teams to participate in the tests more successful for longer while still setting benchmarks in the advanced tests.

\subsection{Replacement of the RoCKIn Object Set with the new Advanced Set}

The objects from previous years technical challenge "Real Object Test" have been added to the new Advanced Object set. This replaces the outdated RoCKIn object set for this season and onwards.

\subsection{Introduction of April Tagged Objects}

With increasing requirements for object recognition due to the introduction of arbitrary surfaces and more difficult objects, the object detection is very critical for teams to being able to perform tasks in the league. To relax the initial boundary of this task, an option to replace the real objects with April tagged cubes has been introduced. Teams can then focus on other aspects of the league (navigation, task optimization, etc.) while being able to participate in the league successfully. As the final benchmark still foresees actual detection and recognition of real objects, a penalty is applied when a team uses this simplification.

\subsection{Clarifications for successful manipulation}

Some unclear situations were adressed with some clarifications regarding successful object placement on the table, in containers and on precise placement tiles.

\subsection{Referee Briefing}

An additional organization slot was added to the schedule that shall be used to brief the referees of each team for the upcoming competition tests. Only briefed refs are allowed to judge the future runs.
Teams are asked to send at least two members to the briefing, if possible.

\subsection{Minimum Ref Number}

A lower boundary of 4 for the amount of refs required for "fair" judgement was added.
In case of only a few attending teams and therefore lower referee numbers,
TC members will assist in the judging process.

\subsection{Technical Challenges}

The technical challenge "Real Object Test" has been removed as the new object set is used in the benchmark tests.



\section{Season 2022}

\subsection{General Changes for 2022}

In 2021, the first virtual robocup was held online via discord, zoom and youtube live.
As a lot of new teams came into the league which never experienced a in-person robocup,
it came clear that the rulebook was missing out on specific information and rule definitions.
A lot of the rules have been habit and spoken agreements during the years,
which included crucial elements such as handling of robot collisions, repeating runs,
on-site competition organization and more.

This years rulebook tries to clarify all these rules, while also releasing some constraints 
while enforcing more robot safety. In the following, 
some rule summaries are being made about the following chapters.
Please carefully read the paragraphs anyways.

\subsection{Chapter 2: League Organization}

Discord has proven to be a very useful tool to organize 
robocups as it allows teams to communicate. All participating teams should join our Discord to keep up on news and announcements.

\subsection{Chapter 3: Robot Rules}

The size constraints were relaxed to allow more chassis types.
However, certain safety requirements must be met to participate in the competition.

\subsection{Chapter 3: Environment Specification}

The requirements for a robot to autonomously navigate the arena have been specified. Robots must fit those specs to participate.

The arena elements and their role in the competition are defined, declaring a task location a service area.
Robots must perform various manipulation tasks at service areas, mostly by handling a set of objects.



\subsection{Chapter 4: The Competition}

The on-site process is defined and explained, most importantly the rules and schedule of runs. Robots must autonomously perform a set of tasks at different service areas, where the exact task definition is defined by so called tests.

Each team has one performance slot for each test type (currently 7+1), with the option to repeat a lower-scoring test once in a later timeslot if possible.
A performance slot consists of a prep phase, a run phase and the end phase, where the performing team prepares and executes their run and then gets their performance then evaluated by the refs.

\subsection{Chapter 5: Test Definition}

Clarification of the basic manipulation, basic transportation, precise placement and rotating table test.


\subsection{Chapter 6: Scoring Adjustments}

The scoring for successful task execution and errors has been updated to ease the competition for newer teams while keeping the ambitions for more difficult tasks with boni.

\subsection{Chapter 7: Virtual Cups}

Taking the rules from 2021s virtual cup, 
requirements for arena setups at home are defined to allow teams to participate in their own laboratory.
They must livestream their test performance with a professional camera setup to allow refs worldwide to evaluate the performance.

\subsection{Chapter 8: Technical Challenges}

Three new technical challenges have been introduced to help evolve the league and the scientific challenges. 
The exact specification of the individual tests is yet to be made.
	
